% =============================================================================
% Lecture 12: Capstone & Applications
% BSAD 8310: Business Forecasting | University of Nebraska at Omaha
% =============================================================================
\documentclass[aspectratio=169, 10pt]{beamer}
% =============================================================================
% header.tex — BSAD 8310: Business Forecasting
% University of Nebraska at Omaha
% Beamer theme: UNO-branded, clean, professional
% =============================================================================

% ----------------------------- BEAMER THEME ----------------------------------
\usetheme{default}
\useinnertheme{rectangles}

% ----------------------------- UNO COLOR PALETTE -----------------------------
\definecolor{unoblue}{HTML}{005CA9}
\definecolor{unored}{HTML}{E41C38}
\definecolor{unogray}{HTML}{525252}
\definecolor{unogreen}{HTML}{15803d}
\definecolor{unolightblue}{HTML}{E8F0FA}
\definecolor{unolightred}{HTML}{FDECEA}
\definecolor{unolightgreen}{HTML}{F0FAF4}
\definecolor{unowhite}{HTML}{FFFFFF}

% Apply UNO colors to Beamer structure
\setbeamercolor{structure}{fg=unoblue}
\setbeamercolor{palette primary}{bg=unoblue, fg=white}
\setbeamercolor{palette secondary}{bg=unoblue!80!black, fg=white}
\setbeamercolor{palette tertiary}{bg=unoblue!60!black, fg=white}
\setbeamercolor{frametitle}{bg=unoblue, fg=white}
\setbeamercolor{frametitle right}{bg=unoblue!80!black}
\setbeamercolor{title}{fg=unoblue}
\setbeamercolor{subtitle}{fg=unogray}
\setbeamercolor{author in head/foot}{bg=unoblue, fg=white}
\setbeamercolor{title in head/foot}{bg=unoblue!80, fg=white}
\setbeamercolor{date in head/foot}{bg=unoblue!60, fg=white}
\setbeamercolor{page number in head/foot}{bg=unoblue!60, fg=white}
\setbeamercolor{block title}{bg=unoblue, fg=white}
\setbeamercolor{block body}{bg=unolightblue}
\setbeamercolor{block title alerted}{bg=unored, fg=white}
\setbeamercolor{block body alerted}{bg=unolightred}
\setbeamercolor{block title example}{bg=unogreen, fg=white}
\setbeamercolor{block body example}{bg=unolightgreen}
\setbeamercolor{itemize item}{fg=unoblue}
\setbeamercolor{itemize subitem}{fg=unored}
\setbeamercolor{enumerate item}{fg=unoblue}
\setbeamercolor{enumerate subitem}{fg=unored}
\setbeamercolor{alerted text}{fg=unored}

% ----------------------------- FONTS -----------------------------------------
\usefonttheme{professionalfonts}
\usefonttheme[onlymath]{serif}       % serif math; sans-serif text
\setbeamerfont{frametitle}{size=\large, series=\bfseries}
\setbeamerfont{title}{size=\LARGE, series=\bfseries}
\setbeamerfont{subtitle}{size=\large}
\setbeamerfont{block title}{size=\normalsize, series=\bfseries}
\setbeamerfont{footline}{size=\tiny}

% ----------------------------- LAYOUT ----------------------------------------
\setbeamersize{text margin left=0.5cm, text margin right=0.5cm}
\setbeamertemplate{navigation symbols}{}   % remove navigation buttons
\setbeamertemplate{itemize items}[circle]
\setbeamertemplate{enumerate items}[default]

% Custom footline: [Course] [Title] [Page/Total]
\setbeamertemplate{footline}{%
  \leavevmode%
  \hbox{%
    \begin{beamercolorbox}[wd=.33\paperwidth, ht=2.5ex, dp=1ex, left, leftskip=4pt]
      {author in head/foot}%
      \usebeamerfont{author in head/foot}\insertshortauthor
    \end{beamercolorbox}%
    \begin{beamercolorbox}[wd=.34\paperwidth, ht=2.5ex, dp=1ex, center]
      {title in head/foot}%
      \usebeamerfont{title in head/foot}\insertshorttitle
    \end{beamercolorbox}%
    \begin{beamercolorbox}[wd=.33\paperwidth, ht=2.5ex, dp=1ex, right, rightskip=4pt]
      {date in head/foot}%
      \usebeamerfont{date in head/foot}%
      \insertframenumber{} / \inserttotalframenumber
    \end{beamercolorbox}%
  }%
  \vskip0pt%
}

% Frametitle with thin accent line
\setbeamertemplate{frametitle}{%
  \vskip0.1cm
  \insertframetitle
  \vskip0.05cm
  \color{unored}\rule{\textwidth}{0.5pt}
}

% Title page
\setbeamertemplate{title page}{%
  \vfill
  \begin{center}
    {\color{unoblue}\rule{\textwidth}{2pt}}\\[0.3cm]
    {\usebeamerfont{title}\usebeamercolor[fg]{title}\inserttitle}\\[0.2cm]
    {\usebeamerfont{subtitle}\usebeamercolor[fg]{subtitle}\insertsubtitle}\\[0.3cm]
    {\color{unored}\rule{\textwidth}{0.5pt}}\\[0.4cm]
    {\small\insertauthor}\\[0.1cm]
    {\small\insertinstitute}\\[0.1cm]
    {\small\insertdate}
  \end{center}
  \vfill
}

% ----------------------------- PACKAGES --------------------------------------

% Math
\usepackage{amsmath}
\usepackage{amssymb}
\usepackage{mathtools}
\usepackage{bm}                    % bold math symbols

% Graphics & color
\usepackage{graphicx}
\usepackage{xcolor}
\usepackage{tikz}
\usetikzlibrary{arrows.meta, positioning, shapes, fit, backgrounds, calc}
\usepackage{pgfplots}
\pgfplotsset{compat=1.18}

% Tables
\usepackage{booktabs}
\usepackage{array}
\usepackage{multirow}
\usepackage{tabularx}

% Typography
\usepackage{microtype}
\usepackage{url}
\usepackage{hyperref}
\hypersetup{colorlinks=true, linkcolor=unoblue, urlcolor=unoblue, citecolor=unogray}

% Code listings (no shell-escape required)
\usepackage{listings}
\lstset{
  language=Python,
  basicstyle=\ttfamily\footnotesize,
  keywordstyle=\color{unoblue}\bfseries,
  stringstyle=\color{unogreen},
  commentstyle=\color{unogray}\itshape,
  numberstyle=\tiny\color{unogray},
  breaklines=true,
  showstringspaces=false,
  frame=single,
  rulecolor=\color{unogray!40},
  backgroundcolor=\color{unogray!5},
  xleftmargin=0.5em,
  xrightmargin=0.5em,
}

% Bibliography
\usepackage[backend=bibtex, style=authoryear, maxcitenames=2]{biblatex}
\addbibresource{../Bibliography_base.bib}

% Colored text helpers
\usepackage{tcolorbox}
\tcbuselibrary{skins, breakable, listingsutf8}

% ----------------------------- CUSTOM ENVIRONMENTS ---------------------------

% keybox: UNO-blue background — for key results, formulas, takeaways
\newtcolorbox{keybox}{
  enhanced,
  colback=unoblue,
  colframe=unoblue!80!black,
  coltitle=white,
  coltext=white,
  fonttitle=\bfseries,
  boxrule=0pt,
  arc=3pt,
  left=4pt, right=4pt, top=3pt, bottom=3pt,
}

% definitionbox: blue left-rule with title — for formal definitions
\newtcolorbox{definitionbox}[1]{
  enhanced,
  title={#1},
  colback=unolightblue,
  colframe=unoblue,
  coltitle=unoblue,
  fonttitle=\bfseries,
  boxrule=0pt,
  leftrule=3pt,
  arc=0pt,
  left=4pt, right=4pt, top=3pt, bottom=3pt,
}

% warningbox: red-accent — for pitfalls, assumption violations, common errors
\newtcolorbox{warningbox}{
  enhanced,
  colback=unolightred,
  colframe=unored,
  coltitle=white,
  fonttitle=\bfseries,
  boxrule=0pt,
  leftrule=3pt,
  arc=0pt,
  left=4pt, right=4pt, top=3pt, bottom=3pt,
}

% examplebox: green-accent with title — for worked examples, business applications
\newtcolorbox{examplebox}[1]{
  enhanced,
  title={#1},
  colback=unolightgreen,
  colframe=unogreen,
  coltitle=unogreen,
  fonttitle=\bfseries,
  boxrule=0pt,
  leftrule=3pt,
  arc=0pt,
  left=4pt, right=4pt, top=3pt, bottom=3pt,
}

% ----------------------------- MATH SHORTCUTS --------------------------------
\newcommand{\E}{\mathbb{E}}
\newcommand{\Var}{\operatorname{Var}}
\newcommand{\Cov}{\operatorname{Cov}}
\newcommand{\Corr}{\operatorname{Corr}}
\newcommand{\MSE}{\operatorname{MSE}}
\newcommand{\RMSE}{\operatorname{RMSE}}
\newcommand{\MAE}{\operatorname{MAE}}
\newcommand{\MASE}{\operatorname{MASE}}
\newcommand{\yhat}{\hat{y}}
\newcommand{\bhat}{\hat{\beta}}
\newcommand{\eps}{\varepsilon}
\newcommand{\given}{\,|\,}

% ----------------------------- SLIDE HELPERS ---------------------------------
% Section title slide (call at start of each section)
\newcommand{\sectionslide}[2]{%
  \begin{frame}
    \vfill
    \begin{center}
      {\color{unoblue}\rule{0.6\textwidth}{2pt}}\\[0.4cm]
      {\Large\bfseries\color{unoblue} #1}\\[0.2cm]
      {\normalsize\color{unogray} #2}\\[0.4cm]
      {\color{unored}\rule{0.6\textwidth}{1pt}}
    \end{center}
    \vfill
  \end{frame}
}

% Muted text
\newcommand{\muted}[1]{{\color{unogray}#1}}

% Key term
\newcommand{\key}[1]{{\color{unoblue}\textbf{#1}}}

% Positive / negative annotations
\newcommand{\pos}[1]{{\color{unogreen}#1}}
\newcommand{\negc}[1]{{\color{unored}#1}}


\title{Lecture 12: Capstone \& Applications}
\subtitle{Synthesizing the Forecasting Toolkit}
\author{BSAD 8310: Business Forecasting}
\institute{University of Nebraska at Omaha}
\date{Spring 2026}

\begin{document}

% --- Title frame --------------------------------------------------------------
\begin{frame}
  \titlepage
\end{frame}

% --- Outline ------------------------------------------------------------------
\begin{frame}{Outline}
  \tableofcontents
\end{frame}

% =============================================================================
\section{Course Synthesis}
% =============================================================================

\sectionslide{Course Synthesis}{%
  Twelve lectures, two frameworks, one decision: which tool fits this problem?}

% --- Slide: L01–L11 Two Parts -------------------------------------------------
\begin{frame}{L01--L11: A Two-Part Toolkit}
  \begin{columns}[T]
    \column{0.48\textwidth}
      \textbf{Part I: Classical Forecasting (L01--L06)}
      \begin{itemize}\small
        \item \textbf{L01} Benchmarks and evaluation discipline
        \item \textbf{L02} Regression-based forecasting
        \item \textbf{L03} Exponential smoothing (ETS)
        \item \textbf{L04} ARIMA and Box-Jenkins workflow
        \item \textbf{L05} Multivariate: VAR, ARIMAX, cointegration
        \item \textbf{L06} Forecast evaluation, DM test, combination
      \end{itemize}
    \column{0.48\textwidth}
      \textbf{Part II: Machine Learning (L07--L11)}
      \begin{itemize}\small
        \item \textbf{L07} Bias-variance, train/val/test, CV
        \item \textbf{L08} Regularization: LASSO, Ridge, Elastic Net
        \item \textbf{L09} Tree methods: Random Forests, XGBoost
        \item \textbf{L10} Neural networks: LSTM, attention
        \item \textbf{L11} Feature engineering and pipeline design
      \end{itemize}
  \end{columns}
  \vspace{0.2cm}
  \begin{keybox}
    {\small \textbf{Capstone question:} Given a new forecasting problem,
    which part of the toolkit do you reach for first --- and how do you
    decide when to switch?}
  \end{keybox}
\end{frame}

% --- Slide: Decision Framework ------------------------------------------------
\begin{frame}{The Forecasting Decision Framework}
  \begin{definitionbox}{Five Questions Before Fitting Any Model}
    {\small
    \begin{enumerate}\setlength{\itemsep}{3pt}
      \item \textbf{How much data?}
            $n < 200$: classical methods are more reliable;
            $n \geq 200$: either framework is viable.
      \item \textbf{How many predictors?}
            $k < 10$: ARIMAX or VAR; $k \geq 10$: regularization or trees.
      \item \textbf{Is the pattern regular?}
            Strong, stable seasonality $\Rightarrow$ SARIMA/ETS are competitive.
      \item \textbf{Interpretability required?}
            Yes $\Rightarrow$ LASSO or SARIMA; No $\Rightarrow$ XGBoost or LSTM.
      \item \textbf{Refit cadence?}
            Weekly refit $\Rightarrow$ prefer simpler models; Monthly refit $\Rightarrow$ ML feasible.
    \end{enumerate}
    }
  \end{definitionbox}
  \vspace{0.1cm}
  \muted{\footnotesize\itshape
    No single model wins every case. The framework replaces intuition with
    discipline. Apply it \emph{before} fitting.}\\[3pt]
  \muted{\footnotesize\itshape Q5 asks about refit cadence.
    If an economic shock makes weekly refit feasible for the first time,
    does that change which model you choose --- or only how you maintain it?}
\end{frame}

% --- Slide: Five Common Failure Modes -----------------------------------------
\begin{frame}{Five Common Failure Modes}
  \begin{warningbox}
    {\small
    \begin{enumerate}\setlength{\itemsep}{4pt}
      \item \textbf{Data leakage} --- features using future information
            inflate in-sample accuracy and collapse out-of-sample.
            (Fix: \texttt{.shift(1)} before every rolling window.)
      \item \textbf{In-sample evaluation only} --- training RMSE is not
            forecast accuracy. Always evaluate on a held-out test set.
      \item \textbf{Wrong metric for the cost structure} --- MAPE fails
            near zero; RMSE penalizes outliers heavily. Match metric
            to business consequences. (L01, L06)
      \item \textbf{No statistical test for differences} --- a lower RMSE
            may be noise. Report Diebold--Mariano $p$-values. (L06)
      \item \textbf{Point forecast without uncertainty} --- A forecast
            without a prediction interval is not actionable for most
            business decisions. Report intervals.
    \end{enumerate}
    }
  \end{warningbox}
\end{frame}

% =============================================================================
\section{Combining and Testing Forecasts}
% =============================================================================

\sectionslide{Combining and Testing Forecasts}{%
  Equal-weight combination is the benchmark that beats most individual models.}

% --- Slide: Forecast Combination ----------------------------------------------
\begin{frame}{Forecast Combination: The Robust Baseline}
  \muted{\footnotesize\parencite{BatesGranger1969,Timmermann2006,StockWatson2004}}
  \vspace{0.1cm}

  \textbf{Why combination works:} individual models capture different
  features of the DGP. Combining reduces variance without increasing bias.\\[4pt]

  \begin{columns}[T]
    \column{0.54\textwidth}
      \begin{keybox}
        {\small \textbf{RSXFS result:}\\[3pt]
        LSTM alone: RMSE $= 1{,}920$\\
        XGBoost alone: RMSE $= 2{,}050$\\
        SARIMA alone: RMSE $= 2{,}840$\\[4pt]
        \textbf{Equal-weight (SARIMA $+$ XGBoost $+$ LSTM):}\\
        RMSE $= 2{,}080$ --- within 9\% of LSTM,
        lower variance, simpler to maintain.}
      \end{keybox}
    \column{0.42\textwidth}
      \begin{examplebox}{M4 Competition}
        {\small The top-ranked hybrid (ES-RNN) used
        combination of exponential smoothing with
        an RNN internally.\\[3pt]
        Pure ML without combination ranked
        \emph{lower} than Theta (classical) on
        short series. \muted{\footnotesize\parencite{Makridakis2020}}}
      \end{examplebox}
  \end{columns}
\end{frame}

% --- Slide: DM Test -----------------------------------------------------------
\begin{frame}{DM Test Across the Leaderboard}
  \muted{\footnotesize\parencite{Diebold1995,Harvey1997}}
  \vspace{0.05cm}

  \muted{\footnotesize\itshape (Recall from L06:
    $DM = \bar{d}/\widehat{\operatorname{se}}(\bar{d}) \sim \mathcal{N}(0,1)$;
    $d_t = e_{1t}^2 - e_{2t}^2$. Harvey--Leybourne--Newbold correction applied;
    $n = 48$ months.)}
  \vspace{0.05cm}

  {\small Pairwise Diebold--Mariano test results on RSXFS (walk-forward errors,
  HAC standard errors). $\star p < 0.05$;\; $\star\star p < 0.01$;\;
  $\star\star\star p < 0.001$;\; n.s.\ not significant.}
  \vspace{0.1cm}

  \begin{center}
  {\footnotesize
  \begin{tabular}{lp{1.7cm}p{1.7cm}p{1.7cm}p{1.7cm}}
    \toprule
    & \textbf{vs.\ SARIMA} & \textbf{vs.\ Elastic Net} & \textbf{vs.\ RF} & \textbf{vs.\ XGBoost} \\
    \midrule
    Elastic Net  & $\star\star\star$ & ---              &                   & \\
    Random Forest & $\star\star\star$ & $\star\star$     & ---               & \\
    XGBoost      & $\star\star\star$ & $\star\star\star$ & $\star$           & --- \\
    LSTM         & $\star\star\star$ & $\star\star\star$ & $\star\star$      & n.s. \\
    Combination  & $\star\star\star$ & $\star\star\star$ & $\star$           & n.s. \\
    \bottomrule
  \end{tabular}
  }
  \end{center}

  \vspace{0.1cm}
  \begin{keybox}
    {\small LSTM vs.\ XGBoost: \textbf{not significant} (n.s.).
    The RMSE gap ($1{,}920$ vs.\ $2{,}050$) does not clear the DM
    threshold. Report $p$-values, not just RMSE gaps.}
  \end{keybox}
\end{frame}

% =============================================================================
\section{RSXFS Final Leaderboard}
% =============================================================================

\sectionslide{RSXFS Final Leaderboard}{%
  Eleven methods, one dataset, a clear pattern.}

% --- Slide: Full Leaderboard --------------------------------------------------
\begin{frame}{All 11 Methods on RSXFS}
  {\footnotesize Test-set results (2020--2023), walk-forward evaluation.
  ML models use 36-feature set ($\dagger$) from L11.}
  \vspace{0.1cm}
  \begin{center}
  {\footnotesize
  \begin{tabular}{llrr}
    \toprule
    \textbf{Lecture} & \textbf{Model} & \textbf{RMSE} & \textbf{MAE} \\
    \midrule
    L01 & Seasonal Na\"{i}ve (benchmark) & 4{,}210 & 3{,}120 \\
    L03 & ETS (auto-AICc)               & 2{,}890 & 2{,}150 \\
    L03 & Holt-Winters (add.)           & 2{,}950 & 2{,}190 \\
    L04 & SARIMA(1,1,1)(1,1,1)$_{12}$   & 2{,}840 & 2{,}100 \\
    L05 & ARIMAX ($+$ sentiment index)  & 2{,}780 & 2{,}060 \\
    L08 & Elastic Net$^\dagger$         & 2{,}410 & 1{,}800 \\
    L08 & Ridge$^\dagger$               & 2{,}460 & 1{,}830 \\
    L09 & Random Forest$^\dagger$       & 2{,}210 & 1{,}640 \\
    L09 & XGBoost$^\dagger$             & 2{,}050 & 1{,}510 \\
    L10 & LSTM (2-layer, $T=24$)$^\dagger$ & 1{,}920 & 1{,}410 \\
    L06 & Equal-weight combination      & 2{,}080 & 1{,}530 \\
    \bottomrule
  \end{tabular}
  }
  \end{center}
  \vspace{0.05cm}
  \muted{\footnotesize\itshape
    Equal-weight combination (2,080) beats XGBoost (2,050) on MAE and
    nearly matches LSTM --- at a fraction of the deployment complexity.}
\end{frame}

% --- Slide: DM Significance ---------------------------------------------------
\begin{frame}{What the DM Test Tells Us}
  \begin{columns}[T]
    \column{0.54\textwidth}
      \textbf{Statistically confirmed improvements:}
      \begin{itemize}\small
        \item All ML methods $\gg$ SARIMA ($p < 0.001$)
        \item XGBoost $>$ Elastic Net ($p < 0.001$)
        \item RF $>$ Elastic Net ($p < 0.01$)
        \item XGBoost $>$ RF ($p < 0.05$) --- marginal
      \end{itemize}
      \vspace{0.2cm}
      \textbf{Not statistically confirmed:}
      \begin{itemize}\small
        \item LSTM vs.\ XGBoost (n.s.)
        \item Combination vs.\ XGBoost (n.s.)
        \item Ridge vs.\ Elastic Net (n.s.)
      \end{itemize}
    \column{0.42\textwidth}
      \begin{keybox}
        {\small \textbf{Practical decision rule:}\\[3pt]
        If two models are DM-indistinguishable,
        choose the \emph{simpler} one.\\[4pt]
        LSTM $\approx$ XGBoost (DM n.s.) $\Rightarrow$
        prefer XGBoost: fewer hyperparameters,
        faster refit, more interpretable
        feature importance.}
      \end{keybox}
  \end{columns}
\end{frame}

% --- Slide: Recommended Pipeline ----------------------------------------------
\begin{frame}{Recommended Deployment Pipeline for RSXFS}
  \begin{keybox}
    {\small \textbf{Five-step production pipeline:}
    \begin{enumerate}\setlength{\itemsep}{3pt}
      \item \textbf{Fit SARIMA} as monitoring anchor and classical baseline.
            Refit monthly on expanding window.
      \item \textbf{Fit XGBoost} with 36-feature set.
            Retrain monthly; monitor feature drift.
      \item \textbf{Form equal-weight combination}:
            $\hat{y}_t = \tfrac{1}{2}(\hat{y}_t^{\text{SARIMA}}
            + \hat{y}_t^{\text{XGB}})$.
      \item \textbf{Monitor quarterly}: run DM test on trailing 12 months.
            If ML no longer improves on SARIMA ($p > 0.10$), revert to
            SARIMA until data accumulates.
      \item \textbf{Report intervals}: 80\% and 95\% prediction intervals
            from bootstrap over walk-forward residuals.
    \end{enumerate}
    }
  \end{keybox}
  \muted{\footnotesize\itshape
    Combination RMSE (2,080) vs.\ XGBoost alone (2,050): the 30-unit RMSE
    gain from LSTM does not justify the added deployment complexity for
    most business settings.}
\end{frame}

% =============================================================================
\section{Case Study: Utility Demand Forecasting}
% =============================================================================

\sectionslide{Case Study: Utility Demand Forecasting}{%
  A different domain to stress-test the decision framework.}

% --- Slide: Business Problem --------------------------------------------------
\begin{frame}{The Business Problem: Residential Natural Gas Demand}
  \begin{columns}[T]
    \column{0.54\textwidth}
      \textbf{Dataset:} Monthly U.S.\ residential natural gas consumption
      (RESGAS, EIA/FRED series \texttt{NGRESCON}).\\[4pt]
      \begin{itemize}\small
        \item \textbf{Period:} Jan 2005 -- Dec 2023 ($n = 228$)
        \item \textbf{Train:} 2005--2019; \textbf{Test:} 2020--2023
        \item \textbf{Seasonality:} extreme (6$\times$ winter/summer ratio),
              virtually no AR structure beyond $m = 12$
        \item \textbf{Trend:} slow decline (efficiency gains)
      \end{itemize}
      \vspace{0.1cm}
      \textbf{Business stake:}
      {\small Natural gas procurement, pipeline capacity planning,
      and hedging contracts require 12-month-ahead forecasts.
      Errors translate directly to over-purchase costs or supply shortfalls.}
    \column{0.42\textwidth}
      \begin{warningbox}
        {\small \textbf{Test period challenge:}
        COVID-19 (2020) shifted residential
        usage patterns. The test period
        is adversarial --- an honest
        stress test for all methods.}
      \end{warningbox}
      \vspace{0.2cm}
      \muted{\footnotesize\itshape
        Contrast with RSXFS: retail sales are
        driven by consumer demand shocks;
        natural gas demand is driven by weather
        and long-term efficiency trends.}
  \end{columns}
\end{frame}

% --- Slide: Applying the Framework --------------------------------------------
\begin{frame}{Applying the Decision Framework to RESGAS}
  {\small Walk through the five questions \emph{before} fitting any model.}
  \vspace{0.1cm}
  \begin{center}
  {\small
  \begin{tabular}{clp{5.2cm}l}
    \toprule
    \textbf{Q} & \textbf{Question} & \textbf{Answer for RESGAS} & \textbf{Implication} \\
    \midrule
    1 & Series length? & 228 months & Either framework \\
    2 & Predictor count? & $\leq 5$ core features & Classical competitive \\
    3 & Seasonality regular? & \textbf{Yes} --- almost purely sinusoidal & SARIMA/ETS strong \\
    4 & Interpretability? & Regulatory context & Prefer classical \\
    5 & Refit cadence? & Monthly feasible & ML possible \\
    \bottomrule
  \end{tabular}
  }
  \end{center}
  \vspace{0.15cm}
  \begin{keybox}
    {\small \textbf{Framework prediction} (before fitting):
    SARIMA and ETS will be competitive with ML.
    Feature engineering will add less value than on RSXFS.
    Prefer SARIMA for interpretability and audit compliance.}
  \end{keybox}
\end{frame}

% --- Slide: RESGAS Leaderboard ------------------------------------------------
\begin{frame}{RESGAS Leaderboard: Framework Prediction vs.\ Results}
  \begin{columns}[T]
    \column{0.46\textwidth}
      {\footnotesize Test-set RMSE (2020--2023,
      walk-forward, units: billion cubic feet/month):}
      \vspace{0.1cm}
      \begin{center}
      {\footnotesize
      \begin{tabular}{lr}
        \toprule
        \textbf{Model} & \textbf{RMSE} \\
        \midrule
        Seasonal Na\"{i}ve  & 12{,}400 \\
        ETS                  & 4{,}650 \\
        \textbf{SARIMA}      & \textbf{4{,}200} \\
        Elastic Net          & 5{,}800 \\
        Random Forest        & 5{,}100 \\
        XGBoost              & 4{,}900 \\
        LSTM                 & 4{,}750 \\
        \bottomrule
      \end{tabular}
      }
      \end{center}
    \column{0.50\textwidth}
      \begin{examplebox}{Finding}
        {\small \textbf{SARIMA wins} (RMSE = 4,200) --- the decision
        framework correctly predicted the outcome before a single
        model was fitted.\\[4pt]
        ML methods (LSTM best at 4,750) do not overcome the
        regular seasonality that SARIMA captures exactly.\\[4pt]
        \textbf{Contrast with RSXFS:}
        LSTM = 1,920 vs.\ SARIMA = 2,840.
        The feature gap from L11 exists only
        when the series has exploitable nonlinear structure.}
      \end{examplebox}
  \end{columns}
\end{frame}

% --- Slide: What Transferred --------------------------------------------------
\begin{frame}{What Transferred, What Did Not}
  \begin{columns}[T]
    \column{0.48\textwidth}
      \textbf{Transferred from RSXFS workflow:}
      \begin{itemize}\small
        \item Walk-forward CV discipline (no shuffling)
        \item Out-of-sample test set (held out throughout)
        \item DM test for statistical significance
        \item Forecast combination as a fallback
        \item Prediction interval reporting
      \end{itemize}
    \column{0.48\textwidth}
      \textbf{Did not transfer:}
      \begin{itemize}\small
        \item ML's large RMSE advantage over SARIMA\\
              {\footnotesize (regular seasonality negates the feature advantage)}
        \item 36-feature set value\\
              {\footnotesize (gas demand has 5 core drivers, not 36)}
        \item LSTM's edge over XGBoost\\
              {\footnotesize (sequential patterns already captured by lags)}
      \end{itemize}
  \end{columns}
  \vspace{0.2cm}
  \begin{keybox}
    {\small \textbf{The decision framework worked.} Applying five questions
    before fitting directed resources toward SARIMA, which won.
    The framework is the product --- not any specific model.}
  \end{keybox}
\end{frame}

% =============================================================================
\section{Communication and Deployment}
% =============================================================================

\sectionslide{Communication and Deployment}{%
  A forecast without uncertainty is just a guess in formal notation.}

% --- Slide: Presenting to Stakeholders ----------------------------------------
\begin{frame}{Presenting Forecasts to Stakeholders}
  \begin{columns}[T]
    \column{0.50\textwidth}
      \textbf{What to include:}
      \begin{itemize}\small
        \item Point forecast with 80\% and 95\%
              prediction intervals
        \item Model used, training window, last
              refit date
        \item RMSE on most recent test period
        \item Known limitations and assumption violations
        \item Update cadence and trigger for review
      \end{itemize}
      \vspace{0.1cm}
      \textbf{What to exclude:}
      \begin{itemize}\small
        \item In-sample $R^2$ or training RMSE
        \item $p$-values without business interpretation
        \item Decimal precision beyond measurement error
      \end{itemize}
    \column{0.46\textwidth}
      \begin{examplebox}{Stakeholder summary}
        {\small \textbf{RSXFS 12-month forecast}\\
        Point estimate: \$452{,}000M\\
        80\% PI: \$444{,}000M -- \$460{,}000M\\
        95\% PI: \$438{,}000M -- \$466{,}000M\\[3pt]
        Model: XGBoost $+$ SARIMA combination,
        retrained monthly (36 features).\\
        Test RMSE (2020--2023): \$2{,}080M.\\[3pt]
        \textbf{Next update:} April 2026 after
        March data release.}
      \end{examplebox}
  \end{columns}
\end{frame}

% --- Slide: Monitoring --------------------------------------------------------
\begin{frame}{Monitoring and Maintenance}
  \begin{columns}[T]
    \column{0.54\textwidth}
      \textbf{Four monitoring checks:}
      \begin{enumerate}\small\setlength{\itemsep}{4pt}
        \item \textbf{Rolling RMSE:} compute walk-forward
              RMSE each period. Flag if $> 2\times$ historical
              average over three consecutive months.
        \item \textbf{Quarterly DM test:} confirm ML still
              outperforms SARIMA baseline. Revert if $p > 0.10$.
        \item \textbf{Feature drift:} monitor input feature
              distributions. Alert if mean shifts $> 2\sigma$
              from training distribution.
        \item \textbf{Model refresh:} retrain on expanding
              window monthly. Benchmark against prior version
              before deploying.
      \end{enumerate}
    \column{0.42\textwidth}
      \begin{warningbox}
        {\small \textbf{Models degrade silently.}\\[3pt]
        A model that was accurate in 2022 may
        be systematically wrong in 2025 due to
        structural shifts in the economy.\\[3pt]
        A monitoring dashboard is not optional
        --- it is the difference between a
        deployed model and a science project.}
      \end{warningbox}
  \end{columns}
\end{frame}

% =============================================================================
\section{Takeaways and References}
% =============================================================================

\sectionslide{Takeaways and References}{What we learned and where to go next.}

% --- Slide: Key Takeaways -----------------------------------------------------
\begin{frame}{BSAD 8310 Key Takeaways}
  \begin{keybox}
    \footnotesize
    \begin{enumerate}\setlength{\itemsep}{2pt}
      \item \textbf{Classical methods} (ETS, ARIMA) are competitive for
            regular, short series with few predictors.
            Match the method to the series structure.
      \item \textbf{ML methods} (regularization, trees, LSTM) add value
            when features are rich, series are long, and patterns are
            nonlinear or structural.
      \item \textbf{Forecast combination} (equal-weight) consistently
            matches or beats the best individual model at lower
            deployment variance \parencite{BatesGranger1969,Timmermann2006}.
      \item \textbf{Evaluation discipline} --- walk-forward CV, DM
            significance test, out-of-sample only --- is
            non-negotiable \parencite{Diebold1995}.
      \item \textbf{Feature engineering} often yields larger RMSE gains
            than switching model class. The feature gap dominates the
            model gap \parencite{Makridakis2020}.
      \item \textbf{Communication} determines whether a technically
            correct forecast is actionable.
            Report intervals, not just point estimates.
    \end{enumerate}
  \end{keybox}
  \smallskip
  \muted{\footnotesize\itshape
    This concludes BSAD 8310: Business Forecasting.
    The toolkit, the discipline, and the decision framework are yours.}
\end{frame}

% --- References ---------------------------------------------------------------
\begin{frame}[allowframebreaks]{References}
  \printbibliography[heading=none]
\end{frame}

\end{document}

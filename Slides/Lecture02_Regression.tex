% =============================================================================
% Lecture 02: Regression-Based Forecasting
% BSAD 8310: Business Forecasting
% University of Nebraska at Omaha
% =============================================================================

\documentclass[aspectratio=169, 11pt]{beamer}

% =============================================================================
% header.tex — BSAD 8310: Business Forecasting
% University of Nebraska at Omaha
% Beamer theme: UNO-branded, clean, professional
% =============================================================================

% ----------------------------- BEAMER THEME ----------------------------------
\usetheme{default}
\useinnertheme{rectangles}

% ----------------------------- UNO COLOR PALETTE -----------------------------
\definecolor{unoblue}{HTML}{005CA9}
\definecolor{unored}{HTML}{E41C38}
\definecolor{unogray}{HTML}{525252}
\definecolor{unogreen}{HTML}{15803d}
\definecolor{unolightblue}{HTML}{E8F0FA}
\definecolor{unolightred}{HTML}{FDECEA}
\definecolor{unolightgreen}{HTML}{F0FAF4}
\definecolor{unowhite}{HTML}{FFFFFF}

% Apply UNO colors to Beamer structure
\setbeamercolor{structure}{fg=unoblue}
\setbeamercolor{palette primary}{bg=unoblue, fg=white}
\setbeamercolor{palette secondary}{bg=unoblue!80!black, fg=white}
\setbeamercolor{palette tertiary}{bg=unoblue!60!black, fg=white}
\setbeamercolor{frametitle}{bg=unoblue, fg=white}
\setbeamercolor{frametitle right}{bg=unoblue!80!black}
\setbeamercolor{title}{fg=unoblue}
\setbeamercolor{subtitle}{fg=unogray}
\setbeamercolor{author in head/foot}{bg=unoblue, fg=white}
\setbeamercolor{title in head/foot}{bg=unoblue!80, fg=white}
\setbeamercolor{date in head/foot}{bg=unoblue!60, fg=white}
\setbeamercolor{page number in head/foot}{bg=unoblue!60, fg=white}
\setbeamercolor{block title}{bg=unoblue, fg=white}
\setbeamercolor{block body}{bg=unolightblue}
\setbeamercolor{block title alerted}{bg=unored, fg=white}
\setbeamercolor{block body alerted}{bg=unolightred}
\setbeamercolor{block title example}{bg=unogreen, fg=white}
\setbeamercolor{block body example}{bg=unolightgreen}
\setbeamercolor{itemize item}{fg=unoblue}
\setbeamercolor{itemize subitem}{fg=unored}
\setbeamercolor{enumerate item}{fg=unoblue}
\setbeamercolor{enumerate subitem}{fg=unored}
\setbeamercolor{alerted text}{fg=unored}

% ----------------------------- FONTS -----------------------------------------
\usefonttheme{professionalfonts}
\usefonttheme[onlymath]{serif}       % serif math; sans-serif text
\setbeamerfont{frametitle}{size=\large, series=\bfseries}
\setbeamerfont{title}{size=\LARGE, series=\bfseries}
\setbeamerfont{subtitle}{size=\large}
\setbeamerfont{block title}{size=\normalsize, series=\bfseries}
\setbeamerfont{footline}{size=\tiny}

% ----------------------------- LAYOUT ----------------------------------------
\setbeamersize{text margin left=0.5cm, text margin right=0.5cm}
\setbeamertemplate{navigation symbols}{}   % remove navigation buttons
\setbeamertemplate{itemize items}[circle]
\setbeamertemplate{enumerate items}[default]

% Custom footline: [Course] [Title] [Page/Total]
\setbeamertemplate{footline}{%
  \leavevmode%
  \hbox{%
    \begin{beamercolorbox}[wd=.33\paperwidth, ht=2.5ex, dp=1ex, left, leftskip=4pt]
      {author in head/foot}%
      \usebeamerfont{author in head/foot}\insertshortauthor
    \end{beamercolorbox}%
    \begin{beamercolorbox}[wd=.34\paperwidth, ht=2.5ex, dp=1ex, center]
      {title in head/foot}%
      \usebeamerfont{title in head/foot}\insertshorttitle
    \end{beamercolorbox}%
    \begin{beamercolorbox}[wd=.33\paperwidth, ht=2.5ex, dp=1ex, right, rightskip=4pt]
      {date in head/foot}%
      \usebeamerfont{date in head/foot}%
      \insertframenumber{} / \inserttotalframenumber
    \end{beamercolorbox}%
  }%
  \vskip0pt%
}

% Frametitle with thin accent line
\setbeamertemplate{frametitle}{%
  \vskip0.1cm
  \insertframetitle
  \vskip0.05cm
  \color{unored}\rule{\textwidth}{0.5pt}
}

% Title page
\setbeamertemplate{title page}{%
  \vfill
  \begin{center}
    {\color{unoblue}\rule{\textwidth}{2pt}}\\[0.3cm]
    {\usebeamerfont{title}\usebeamercolor[fg]{title}\inserttitle}\\[0.2cm]
    {\usebeamerfont{subtitle}\usebeamercolor[fg]{subtitle}\insertsubtitle}\\[0.3cm]
    {\color{unored}\rule{\textwidth}{0.5pt}}\\[0.4cm]
    {\small\insertauthor}\\[0.1cm]
    {\small\insertinstitute}\\[0.1cm]
    {\small\insertdate}
  \end{center}
  \vfill
}

% ----------------------------- PACKAGES --------------------------------------

% Math
\usepackage{amsmath}
\usepackage{amssymb}
\usepackage{mathtools}
\usepackage{bm}                    % bold math symbols

% Graphics & color
\usepackage{graphicx}
\usepackage{xcolor}
\usepackage{tikz}
\usetikzlibrary{arrows.meta, positioning, shapes, fit, backgrounds, calc}
\usepackage{pgfplots}
\pgfplotsset{compat=1.18}

% Tables
\usepackage{booktabs}
\usepackage{array}
\usepackage{multirow}
\usepackage{tabularx}

% Typography
\usepackage{microtype}
\usepackage{url}
\usepackage{hyperref}
\hypersetup{colorlinks=true, linkcolor=unoblue, urlcolor=unoblue, citecolor=unogray}

% Code listings (no shell-escape required)
\usepackage{listings}
\lstset{
  language=Python,
  basicstyle=\ttfamily\footnotesize,
  keywordstyle=\color{unoblue}\bfseries,
  stringstyle=\color{unogreen},
  commentstyle=\color{unogray}\itshape,
  numberstyle=\tiny\color{unogray},
  breaklines=true,
  showstringspaces=false,
  frame=single,
  rulecolor=\color{unogray!40},
  backgroundcolor=\color{unogray!5},
  xleftmargin=0.5em,
  xrightmargin=0.5em,
}

% Bibliography
\usepackage[backend=bibtex, style=authoryear, maxcitenames=2]{biblatex}
\addbibresource{../Bibliography_base.bib}

% Colored text helpers
\usepackage{tcolorbox}
\tcbuselibrary{skins, breakable, listingsutf8}

% ----------------------------- CUSTOM ENVIRONMENTS ---------------------------

% keybox: UNO-blue background — for key results, formulas, takeaways
\newtcolorbox{keybox}{
  enhanced,
  colback=unoblue,
  colframe=unoblue!80!black,
  coltitle=white,
  coltext=white,
  fonttitle=\bfseries,
  boxrule=0pt,
  arc=3pt,
  left=4pt, right=4pt, top=3pt, bottom=3pt,
}

% definitionbox: blue left-rule with title — for formal definitions
\newtcolorbox{definitionbox}[1]{
  enhanced,
  title={#1},
  colback=unolightblue,
  colframe=unoblue,
  coltitle=unoblue,
  fonttitle=\bfseries,
  boxrule=0pt,
  leftrule=3pt,
  arc=0pt,
  left=4pt, right=4pt, top=3pt, bottom=3pt,
}

% warningbox: red-accent — for pitfalls, assumption violations, common errors
\newtcolorbox{warningbox}{
  enhanced,
  colback=unolightred,
  colframe=unored,
  coltitle=white,
  fonttitle=\bfseries,
  boxrule=0pt,
  leftrule=3pt,
  arc=0pt,
  left=4pt, right=4pt, top=3pt, bottom=3pt,
}

% examplebox: green-accent with title — for worked examples, business applications
\newtcolorbox{examplebox}[1]{
  enhanced,
  title={#1},
  colback=unolightgreen,
  colframe=unogreen,
  coltitle=unogreen,
  fonttitle=\bfseries,
  boxrule=0pt,
  leftrule=3pt,
  arc=0pt,
  left=4pt, right=4pt, top=3pt, bottom=3pt,
}

% ----------------------------- MATH SHORTCUTS --------------------------------
\newcommand{\E}{\mathbb{E}}
\newcommand{\Var}{\operatorname{Var}}
\newcommand{\Cov}{\operatorname{Cov}}
\newcommand{\Corr}{\operatorname{Corr}}
\newcommand{\MSE}{\operatorname{MSE}}
\newcommand{\RMSE}{\operatorname{RMSE}}
\newcommand{\MAE}{\operatorname{MAE}}
\newcommand{\MASE}{\operatorname{MASE}}
\newcommand{\yhat}{\hat{y}}
\newcommand{\bhat}{\hat{\beta}}
\newcommand{\eps}{\varepsilon}
\newcommand{\given}{\,|\,}

% ----------------------------- SLIDE HELPERS ---------------------------------
% Section title slide (call at start of each section)
\newcommand{\sectionslide}[2]{%
  \begin{frame}
    \vfill
    \begin{center}
      {\color{unoblue}\rule{0.6\textwidth}{2pt}}\\[0.4cm]
      {\Large\bfseries\color{unoblue} #1}\\[0.2cm]
      {\normalsize\color{unogray} #2}\\[0.4cm]
      {\color{unored}\rule{0.6\textwidth}{1pt}}
    \end{center}
    \vfill
  \end{frame}
}

% Muted text
\newcommand{\muted}[1]{{\color{unogray}#1}}

% Key term
\newcommand{\key}[1]{{\color{unoblue}\textbf{#1}}}

% Positive / negative annotations
\newcommand{\pos}[1]{{\color{unogreen}#1}}
\newcommand{\negc}[1]{{\color{unored}#1}}


% ---- Lecture metadata --------------------------------------------------------
\title{Regression-Based Forecasting}
\subtitle{BSAD 8310: Business Forecasting --- Lecture 2}
\author{Department of Economics}
\institute{University of Nebraska at Omaha}
\date{Spring 2026}

% =============================================================================
\begin{document}
% =============================================================================

\begin{frame}
  \titlepage
\end{frame}

% --- Outline -----------------------------------------------------------------
\begin{frame}{Lecture 2: Outline}
  \tableofcontents
\end{frame}

% =============================================================================
\section{From Benchmarks to Regression}
% =============================================================================

\sectionslide{From Benchmarks to Regression}{%
  Na\"{i}ve models ignore information. Regression lets us use it.}

% --- Slide: Limits of benchmarks ---------------------------------------------
\begin{frame}{What Benchmark Models Cannot Do}
  Recall the four benchmarks from Lecture 1:
  \begin{center}
    \begin{tabular}{ll}
      \toprule
      \textbf{Benchmark} & \textbf{Information used} \\
      \midrule
      Na\"{i}ve & Last observed value only \\
      Seasonal na\"{i}ve & Same season, last year only \\
      Historical mean & Unconditional average \\
      Random walk + drift & Last value + average change \\
      \bottomrule
    \end{tabular}
  \end{center}
  \vspace{0.2cm}
  \begin{warningbox}
    None of these benchmarks can use \textbf{leading indicators} ---
    variables that move \emph{before} $y_t$ does.
  \end{warningbox}
  \muted{\footnotesize\itshape Examples: consumer confidence $\to$ retail sales next quarter; interest rates $\to$ housing starts.}
\end{frame}

% --- Slide: What regression adds ---------------------------------------------
\begin{frame}{What Regression-Based Forecasting Adds}
  \begin{columns}[T]
    \begin{column}{0.48\textwidth}
      \textbf{Information we can exploit:}
      \begin{itemize}
        \item \key{Leading indicators}: variables observable at $T$
              that predict $y_{T+h}$
        \item \key{Structural relationships}: economic theory says
              $x_t$ affects $y_t$
        \item \key{Deterministic patterns}: known trend, calendar
              effects, holidays
        \item \key{Own-lags}: past $y$ values predict future $y$
      \end{itemize}
    \end{column}
    \begin{column}{0.48\textwidth}
      \begin{examplebox}{Business Context}
        \begin{itemize}
          \item Retailer: use \emph{ad spend}, \emph{price}, and
                \emph{calendar flags} to forecast demand
          \item Macro analyst: use \emph{leading index} to forecast
                next quarter's GDP
          \item Bank: use \emph{yield curve} to forecast loan defaults
                6 months ahead
        \end{itemize}
      \end{examplebox}
    \end{column}
  \end{columns}
\end{frame}

% =============================================================================
\section{The Regression Model for Forecasting}
% =============================================================================

\sectionslide{The Regression Model for Forecasting}{%
  OLS is not just for causal inference --- it is also a forecasting tool.}

% --- Slide: General setup ----------------------------------------------------
\begin{frame}{General Setup}
  \begin{definitionbox}{Regression Model for Forecasting}
    \[
      y_t \;=\; \mathbf{x}_t^{\prime}\boldsymbol{\beta} + \varepsilon_t,
      \qquad t = 1, \ldots, T
    \]
    \begin{itemize}
      \item $y_t$: scalar response (what we want to forecast)
      \item $\mathbf{x}_t = (1, x_{1t}, \ldots, x_{kt})^{\prime}$:
            $(k+1)$-vector of predictors (including intercept)
      \item $\boldsymbol{\beta} = (\beta_0, \beta_1, \ldots, \beta_k)^{\prime}$:
            unknown coefficients
      \item $\varepsilon_t \sim WN(0, \sigma^2)$: white-noise errors
    \end{itemize}
  \end{definitionbox}
  \vspace{0.2cm}
  \muted{\small \textbf{Notation reminder} (from Lecture 1):
  $\varepsilon_t$ denotes the true model innovation;
  $e_t = y_t - \hat{y}_{t|t-1}$ denotes the realized forecast error.
  These are conceptually distinct.}
\end{frame}

% --- Slide: OLS estimation ---------------------------------------------------
\begin{frame}{OLS Estimation}
  Stack observations into matrices: $\mathbf{y} = (y_1, \ldots, y_T)^{\prime}$,
  $\mathbf{X} = (\mathbf{x}_1, \ldots, \mathbf{x}_T)^{\prime}$ ($T \times (k+1)$).

  \vspace{0.2cm}
  \begin{definitionbox}{OLS Estimator}
    \[
      \hat{\boldsymbol{\beta}}
      \;=\; \arg\min_{\boldsymbol{\beta}} \sum_{t=1}^{T}(y_t - \mathbf{x}_t^{\prime}\boldsymbol{\beta})^2
      \;=\; (\mathbf{X}^{\prime}\mathbf{X})^{-1}\mathbf{X}^{\prime}\mathbf{y}
    \]
  \end{definitionbox}
  \vspace{0.15cm}
  \textbf{Gauss-Markov theorem:} Under classical assumptions, $\hat{\boldsymbol{\beta}}$
  is the \key{Best Linear Unbiased Estimator} (BLUE) \parencite[][Ch.~10]{Hamilton1994}.

  \vspace{0.15cm}
  \textbf{Classical assumptions for valid OLS inference:}
  \begin{enumerate}
    \item $\E[\varepsilon_t \mid \mathbf{X}] = 0$ \hfill (zero conditional mean)
    \item $\E[\varepsilon_t^2 \mid \mathbf{X}] = \sigma^2$ \hfill (homoscedasticity)
    \item $\E[\varepsilon_t \varepsilon_s \mid \mathbf{X}] = 0$ for $t \neq s$ \hfill (no autocorrelation)
  \end{enumerate}
\end{frame}

% --- Slide: Predictor types --------------------------------------------------
\begin{frame}{What Can $\mathbf{x}_{T+h}$ Contain?}
  \begin{keybox}
    \textbf{The forecasting constraint:} to use $\mathbf{x}_{T+h}$ in the forecast
    $\hat{y}_{T+h|T}$, every element of $\mathbf{x}_{T+h}$ must be
    \emph{known} or \emph{forecastable} at time $T$.
  \end{keybox}
  \vspace{0.2cm}
  \begin{columns}[T]
    \begin{column}{0.32\textwidth}
      \textbf{\pos{Type 1: Deterministic}}\\[4pt]
      Always known at $T$:
      \begin{itemize}
        \item Time index: $t, t^2$
        \item Seasonal dummies
        \item Holiday indicators
        \item Trend functions
      \end{itemize}
    \end{column}
    \begin{column}{0.32\textwidth}
      \textbf{\pos{Type 2: Lagged values}}\\[4pt]
      Known when $h \leq$ lag:
      \begin{itemize}
        \item $y_{t-1}, y_{t-2}$
        \item $x_{1,t-1}$ (lagged)
        \item Economic releases
              with delay
      \end{itemize}
    \end{column}
    \begin{column}{0.32\textwidth}
      \textbf{\negc{Type 3: External regressors}}\\[4pt]
      Must be forecast first:
      \begin{itemize}
        \item GDP, CPI forecasts
        \item Competitor prices
        \item Weather forecasts
        \item Budget scenarios
      \end{itemize}
    \end{column}
  \end{columns}
\end{frame}

% --- Slide: Forecast formula -------------------------------------------------
\begin{frame}{The Regression Forecast}
  Given $\hat{\boldsymbol{\beta}}$ estimated on the training set $\{1, \ldots, T\}$:

  \begin{keybox}
    \[
      \hat{y}_{T+h|T} \;=\; \mathbf{x}_{T+h}^{\prime}\hat{\boldsymbol{\beta}}
    \]
    Plug the \emph{known or forecasted} predictor vector $\mathbf{x}_{T+h}$
    into the fitted model.
  \end{keybox}
  \vspace{0.2cm}
  \textbf{Important nuances:}
  \begin{itemize}
    \item $\hat{\boldsymbol{\beta}}$ is estimated only on training data
          --- never on the test set
    \item If $\mathbf{x}_{T+h}$ contains forecasted values, forecast
          uncertainty propagates (widens intervals)
    \item For $h > 1$ with lagged-$y$ predictors, use
          \emph{recursive substitution} (see Section 5)
  \end{itemize}
  \muted{\footnotesize\itshape In Lab 02, $\mathbf{x}_{T+h}$ will contain only deterministic terms (trend index + seasonal dummies), so $\mathbf{x}_{T+h}$ is always known exactly.}
\end{frame}

% =============================================================================
\section{Trend and Seasonality as Regressors}
% =============================================================================

\sectionslide{Trend and Seasonality as Regressors}{%
  Deterministic components require no forecasting: their values are always known at the forecast origin.}

% --- Slide: Linear trend regression ------------------------------------------
\begin{frame}{Model 1: Linear Trend}
  The simplest structural model for a series with an upward or downward drift:
  \begin{definitionbox}{Linear Trend Regression}
    \[
      y_t \;=\; \beta_0 + \beta_1 t + \varepsilon_t,
      \qquad t = 1, 2, \ldots, T
    \]
  \end{definitionbox}
  \vspace{0.1cm}
  \textbf{Forecast:} $\hat{y}_{T+h|T} = \hat{\beta}_0 + \hat{\beta}_1(T+h)$ --- always known.

  \vspace{0.15cm}
  \begin{columns}[T]
    \begin{column}{0.48\textwidth}
      \textbf{When appropriate:}
      \begin{itemize}
        \item Steady linear growth or decline
        \item No pronounced seasonality
        \item Relatively stable variance
      \end{itemize}
    \end{column}
    \begin{column}{0.48\textwidth}
      \begin{warningbox}
        \small
        Linear trend extrapolates indefinitely.
        Long-horizon forecasts assume growth
        continues forever --- verify this is
        substantively reasonable.
      \end{warningbox}
    \end{column}
  \end{columns}
  \muted{\footnotesize\itshape US retail sales grow by roughly 3\% per year. Should you use linear or log-linear trend? What feature of the data would help you decide?}
\end{frame}

% --- Slide: Seasonal dummies -------------------------------------------------
\begin{frame}{Model 2: Seasonal Dummy Variables}
  \begin{definitionbox}{Seasonal Dummy Variables ($m$ seasons)}
    {\small Define $D_{j,t} = 1$ if observation $t$ is in season $j$, else $0$.
    \[
      y_t \;=\; \beta_0 + \sum_{j=2}^{m} \gamma_j D_{j,t} + \varepsilon_t
    \]
    Include $m-1$ dummies; drop one season as the \textbf{base category}.}
  \end{definitionbox}
  \begin{columns}[T]
    \begin{column}{0.55\textwidth}
      \textbf{Interpretation:}
      {\small\begin{itemize}
        \item $\beta_0$: mean level in the base season (e.g., January)
        \item $\gamma_j$: seasonal deviation from the base in season $j$
        \item $\hat{y}_{T+h|T} = \hat{\beta}_0 + \hat{\gamma}_{s(T+h)}$
      \end{itemize}}
    \end{column}
    \begin{column}{0.43\textwidth}
      \begin{warningbox}
        \small
        \textbf{Dummy trap:} Including all $m$ dummies creates
        exact multicollinearity with the intercept.
        Always drop one season.
      \end{warningbox}
    \end{column}
  \end{columns}
\end{frame}

% --- Slide: Trend + seasonality ----------------------------------------------
\begin{frame}{Model 3: Trend + Seasonality}
  \begin{definitionbox}{Trend + Seasonal Regression}
    \[
      y_t \;=\; \underbrace{\beta_0 + \beta_1 t}_{\text{trend}}
             + \underbrace{\sum_{j=2}^{m} \gamma_j D_{j,t}}_{\text{seasonal}}
             + \varepsilon_t
    \]
  \end{definitionbox}
  \textbf{Forecast:}
  $\hat{y}_{T+h|T} = \hat{\beta}_0 + \hat{\beta}_1(T+h) + \hat{\gamma}_{s(T+h)}$ \hfill\muted{(always known)}

  \begin{examplebox}{US Retail Sales (RSXFS)}
    {\small Monthly retail sales exhibit both components:
    \begin{itemize}
      \item Upward trend (growing economy and population)
      \item Strong December spike, January dip (Lab 02 quantifies both)
    \end{itemize}}
  \end{examplebox}
\end{frame}

% --- Slide: What about non-linear trend? ------------------------------------
\begin{frame}{Extensions: Non-Linear Trend}
  \textbf{Polynomial trend:}
  \[
    y_t = \beta_0 + \beta_1 t + \beta_2 t^2 + \varepsilon_t
  \]
  Still OLS --- just add $t^2$ as another predictor column.

  \vspace{0.2cm}
  \textbf{Log-linear trend} (when variance grows with level):
  \[
    \ln y_t = \beta_0 + \beta_1 t + \varepsilon_t
    \quad\Leftrightarrow\quad
    y_t = e^{\beta_0}\, e^{\beta_1 t}\, e^{\varepsilon_t}
  \]
  Forecasts: $\hat{y}_{T+h|T} = \exp\!\bigl(\hat{\beta}_0 + \hat{\beta}_1(T+h)
  + \tfrac{1}{2}\hat{\sigma}^2\bigr)$ \hfill\muted{\small (bias correction)}

  \vspace{0.15cm}
  \begin{keybox}
    \small
    \textbf{Rule of thumb:} If the series grows at a roughly
    \emph{constant percentage rate} (e.g., 3\% per year), use
    log-linear trend. If it grows by a \emph{constant dollar amount},
    use linear trend.
  \end{keybox}
\end{frame}

% =============================================================================
\section{Prediction Intervals}
% =============================================================================

\sectionslide{Prediction Intervals}{%
  A point forecast without uncertainty bounds is an incomplete forecast.}

% --- Slide: Point vs. interval -----------------------------------------------
\begin{frame}{From Point Forecast to Prediction Interval}
  \begin{definitionbox}{95\% Prediction Interval}
    \[
      \hat{y}_{T+h|T} \;\pm\; t_{0.025,\,T-k-1}\;\cdot\;
      \hat{\sigma}_e \sqrt{\underbrace{1}_{\substack{\text{irreducible}\\\text{error}}}
      + \underbrace{\mathbf{x}_{T+h}^{\prime}
      (\mathbf{X}^{\prime}\mathbf{X})^{-1}\mathbf{x}_{T+h}}_{\text{estimation uncertainty}}}
    \]
    where $\hat{\sigma}_e^2 = \frac{1}{T-k-1}\sum_{t=1}^{T}e_t^2$
    is the estimated residual variance.
  \end{definitionbox}
  \vspace{0.1cm}
  \textbf{Two sources of uncertainty in the PI:}
  \begin{itemize}
    \item \textbf{Irreducible:} future shock $\varepsilon_{T+h}$
          has variance $\sigma^2$ \hfill (the ``1'' under the square root)
    \item \textbf{Estimation:} $\hat{\boldsymbol{\beta}} \neq \boldsymbol{\beta}$
          \hfill (the $(X'X)^{-1}$ term; shrinks as $T \to \infty$)
  \end{itemize}
  \muted{\footnotesize\itshape PIs widen for predictions far from the center of the training data — including long-horizon forecasts where $t = T+h$ is far from $\bar{t}$.}
\end{frame}

% --- Slide: PI assumptions and violations ------------------------------------
\begin{frame}{When Prediction Intervals Are Valid}
  The standard PI formula assumes:

  \vspace{0.1cm}
  \begin{columns}[T]
    \begin{column}{0.48\textwidth}
      {\small\begin{enumerate}
        \item $\varepsilon_t \stackrel{iid}{\sim} \mathcal{N}(0,\sigma^2)$
              (normality + homoscedasticity)
        \item No autocorrelation: $\Cov(\varepsilon_t, \varepsilon_s) = 0$
        \item No model misspecification
        \item Predictors $\mathbf{x}_{T+h}$ known exactly
      \end{enumerate}}
    \end{column}
    \begin{column}{0.48\textwidth}
      \begin{warningbox}
        \small
        Violation of any assumption
        \textbf{invalidates} the nominal coverage.
        Always check: residual histogram,
        ACF of residuals, Breusch-Pagan test
        for heteroscedasticity.
        (Lecture 6 covers formal tests.)
      \end{warningbox}
    \end{column}
  \end{columns}
  \vspace{0.15cm}
  \begin{keybox}
    \small
    \textbf{Empirical rule of thumb:} If ACF of residuals shows spikes
    at short lags (1–4), the PI is too narrow. Residual autocorrelation
    is exploitable information --- switch to an AR or ARIMA model.
  \end{keybox}
\end{frame}

% =============================================================================
\section{Autoregressive Models}
% =============================================================================

\sectionslide{Autoregressive Models}{%
  The past values of a series often predict its future. AR models formalize this idea.}

% --- Slide: AR(1) as regression -----------------------------------------------
\begin{frame}{The AR(1) Model}
  An \key{autoregressive model of order 1} is just OLS regression
  with one lag of $y$ as the predictor:

  \begin{definitionbox}{AR(1) Model}
    \[
      y_t \;=\; \phi_0 + \phi_1 y_{t-1} + \varepsilon_t,
      \qquad \varepsilon_t \stackrel{iid}{\sim}(0,\sigma^2)
      \;\muted{\small\text{(normality not required for OLS)}}
    \]
  \end{definitionbox}
  \vspace{0.1cm}
  \textbf{Stationarity condition:} $|\phi_1| < 1$. If $\phi_1 = 1$, the model
  is a \emph{random walk} (non-stationary, covered in Lecture 4).

  \vspace{0.1cm}
  \begin{columns}[T]
    \begin{column}{0.48\textwidth}
      \textbf{One-step forecast:}
      \[
        \hat{y}_{T+1|T} = \hat{\phi}_0 + \hat{\phi}_1 y_T
      \]
    \end{column}
    \begin{column}{0.48\textwidth}
      \textbf{Two-step forecast (recursive):}
      \[
        \hat{y}_{T+2|T} = \hat{\phi}_0 + \hat{\phi}_1 \hat{y}_{T+1|T}
      \]
    \end{column}
  \end{columns}
  \vspace{0.1cm}
  \muted{\footnotesize\itshape As $h \to \infty$, multi-step AR(1) forecasts converge to the unconditional mean $\mu = \phi_0/(1-\phi_1)$ when $|\phi_1|<1$.}
\end{frame}

% --- Slide: AR(p) and lag selection ------------------------------------------
\begin{frame}{The AR($p$) Model and Lag Selection}
  \begin{definitionbox}{AR($p$) Model}
    \[
      y_t \;=\; \phi_0 + \phi_1 y_{t-1} + \cdots + \phi_p y_{t-p} + \varepsilon_t
            \;=\; \phi_0 + \sum_{i=1}^{p} \phi_i y_{t-i} + \varepsilon_t
    \]
  \end{definitionbox}
  \vspace{0.1cm}
  \textbf{Lag order selection:} choose $p$ to minimize an information criterion.

  \vspace{0.1cm}
  \begin{columns}[T]
    \begin{column}{0.48\textwidth}
      \begin{align*}
        \text{AIC} &= -2\hat{\ell} + 2(p+2) \\[4pt]
        \text{BIC} &= -2\hat{\ell} + (p+2)\ln T
      \end{align*}
      \muted{\small$\hat{\ell}$: maximized log-likelihood.
      $p+2$ parameters: $p$ AR coefs + intercept + $\sigma^2$.}
    \end{column}
    \begin{column}{0.48\textwidth}
      \begin{keybox}
        \small
        AIC tends to select more lags
        (asymptotically inconsistent: does not
        recover the true order with probability~1).
        BIC penalizes more and is consistent.
        \textbf{Use BIC} for forecasting in most
        business applications \parencite[][Ch.~5]{BoxJenkins2015}.
      \end{keybox}
    \end{column}
  \end{columns}
  \muted{\footnotesize\itshape BIC selects $p=2$ but the residual ACF shows a significant spike at lag~12. What should you do?}
\end{frame}

% --- Slide: Recursive multi-step forecasts -----------------------------------
\begin{frame}{Multi-Step AR Forecasts: The Recursive Method}
  For an AR($p$) model, $h$-step-ahead forecasts use the
  \key{recursive substitution} method:

  \vspace{0.1cm}
  \[
    \hat{y}_{T+h|T} = \hat{\phi}_0 + \sum_{i=1}^{p} \hat{\phi}_i\, \hat{y}_{T+h-i|T}
  \]
  where $\hat{y}_{T+j|T} = y_{T+j}$ for $j \leq 0$ (observed values).

  \vspace{0.15cm}
  \begin{examplebox}{AR(2), $h=3$ example}
    {\small
    \begin{align*}
      \hat{y}_{T+1|T} &= \hat{\phi}_0 + \hat{\phi}_1 y_T + \hat{\phi}_2 y_{T-1} \\
      \hat{y}_{T+2|T} &= \hat{\phi}_0 + \hat{\phi}_1 \hat{y}_{T+1|T} + \hat{\phi}_2 y_T \\
      \hat{y}_{T+3|T} &= \hat{\phi}_0 + \hat{\phi}_1 \hat{y}_{T+2|T} + \hat{\phi}_2 \hat{y}_{T+1|T}
    \end{align*}}
  \end{examplebox}
  \muted{\footnotesize\itshape Forecast uncertainty compounds with horizon: each substitution propagates estimation error from earlier steps.}
\end{frame}

% --- Slide: AR diagnostics ---------------------------------------------------
\begin{frame}{Diagnosing an AR Model}
  \textbf{How do we know if the AR order is adequate?}

  \vspace{0.1cm}
  \begin{columns}[T]
    \begin{column}{0.48\textwidth}
      \textbf{Residual ACF check:}
      \begin{itemize}
        \item Compute residuals $e_t = y_t - \hat{y}_{t|t-1}$
        \item Plot autocorrelation function (ACF) of residuals
        \item All spikes should fall inside the
              95\% bands: $\pm 1.96/\sqrt{T}$
        \item Significant spike at lag $k$
              $\Rightarrow$ add lag $k$ to the model
      \end{itemize}
    \end{column}
    \begin{column}{0.48\textwidth}
      \begin{warningbox}
        \small
        Residual autocorrelation signals
        information the model has not yet captured.
        A model with autocorrelated residuals
        is \textbf{suboptimal} and its prediction
        intervals are \textbf{invalid}.
        Full treatment: Lecture 4 (ARIMA).
      \end{warningbox}
    \end{column}
  \end{columns}
  \vspace{0.15cm}
  \muted{\footnotesize\itshape Lab 02 plots the residual ACF for the AR($p$) model and checks whether any spikes suggest a higher order is needed.}
\end{frame}

% =============================================================================
\section{Pitfalls}
% =============================================================================

\sectionslide{Pitfalls}{%
  Regression in time series requires extra care. Three failure modes to avoid.}

% --- Slide: Spurious regression ----------------------------------------------
\begin{frame}{Pitfall 1: Spurious Regression}
  \begin{warningbox}
    \textcite{GrangerNewbold1974}: Regressing two \emph{independent}
    random-walk series $y_t$ and $x_t$ yields $R^2 \approx 0.7$
    and apparently significant $t$-statistics ---
    despite no true relationship between them.
  \end{warningbox}
  \vspace{0.1cm}
  \textbf{Why this happens:}
  \begin{itemize}
    \item Both $y_t$ and $x_t$ are $I(1)$ (non-stationary; require one difference
          to become stationary): their levels share a stochastic trend
    \item The usual $t$-distribution critical values are \emph{invalid} under non-stationarity
  \end{itemize}
  \begin{columns}[T]
    \begin{column}{0.55\textwidth}
      \textbf{Diagnostic heuristic:} $R^2 > DW$ statistic suggests
      spurious regression \parencite{GrangerNewbold1974}.
    \end{column}
    \begin{column}{0.43\textwidth}
      \begin{keybox}
        \small
        \textbf{Fix:} Difference the series
        ($\Delta y_t$, $\Delta x_t$) before
        regressing. Or test for
        cointegration (Lecture 5).
      \end{keybox}
    \end{column}
  \end{columns}
\end{frame}

% --- Slide: Overfitting ------------------------------------------------------
\begin{frame}{Pitfall 2: Overfitting}
  \textbf{More predictors $\Rightarrow$ better in-sample fit, but often worse out-of-sample forecast.}

  \vspace{0.1cm}
  \begin{columns}[T]
    \begin{column}{0.48\textwidth}
      \textbf{What happens:}
      \begin{itemize}
        \item OLS fits noise along with signal
        \item High $R^2$ in-sample; large errors out-of-sample
        \item Worse with small $T$, large $k$
        \item Many lags: AR(20) on monthly data
              likely overfits
      \end{itemize}
      \vspace{0.1cm}
      \muted{\small \textbf{Rule of thumb:} Keep $k \ll T$. A common bound: $k \leq T/10$.}
    \end{column}
    \begin{column}{0.48\textwidth}
      \begin{examplebox}{Why This Matters}
        A 12-month history and 11 seasonal
        dummies + intercept = 12 parameters.
        You are fitting the data, not the
        underlying pattern.
        \smallskip

        \muted{\small Regularization (LASSO, Ridge) addresses
        this formally in Lecture 8.}
      \end{examplebox}
    \end{column}
  \end{columns}
\end{frame}

% --- Slide: The regressor availability problem -------------------------------
\begin{frame}{Pitfall 3: The Regressor Availability Problem}
  \begin{warningbox}
    \textbf{Critical constraint:} To forecast $y_{T+h}$ using a regressor
    $x_{T+h}$, you must know (or forecast) $x_{T+h}$ at time $T$.
    Using data that won't be available at the forecast origin is
    \key{data leakage}.
  \end{warningbox}
  \vspace{0.15cm}
  \begin{columns}[T]
    \begin{column}{0.48\textwidth}
      \textbf{\pos{Valid predictors for $y_{T+1}$:}}
      \begin{itemize}
        \item[\pos{$\checkmark$}] $y_T, y_{T-1}, \ldots$ (lagged $y$)
        \item[\pos{$\checkmark$}] $x_{T-1}$ (lagged leading indicator)
        \item[\pos{$\checkmark$}] Seasonal dummy at $T+1$
        \item[\pos{$\checkmark$}] Weather forecast for $T+1$
      \end{itemize}
    \end{column}
    \begin{column}{0.48\textwidth}
      \textbf{\negc{Invalid predictors for $y_{T+1}$:}}
      \begin{itemize}
        \item[\negc{$\times$}] $x_{T+1}$ (contemporaneous, unknown)
        \item[\negc{$\times$}] $y_{T+1}$ (the target itself)
        \item[\negc{$\times$}] Future CPI or GDP
        \item[\negc{$\times$}] Any variable with
              publication lag $< h$
      \end{itemize}
    \end{column}
  \end{columns}
\end{frame}

% =============================================================================
\section{Lecture 2: Key Takeaways}
% =============================================================================

\begin{frame}{Lecture 2: Key Takeaways}
  \begin{keybox}
    \small
    \begin{enumerate}
      \item Regression extends benchmarks by incorporating \textbf{predictors}:
            deterministic trends, seasonal dummies, lagged values,
            and leading indicators.
      \item The OLS forecast is $\hat{y}_{T+h|T} = \mathbf{x}_{T+h}^{\prime}\hat{\boldsymbol{\beta}}$,
            but $\mathbf{x}_{T+h}$ must be \textbf{known or forecastable} at time $T$.
      \item Prediction intervals have two components: irreducible error
            ($\sigma^2$) and estimation uncertainty ($(X'X)^{-1}$ term).
            Valid only under no autocorrelation and homoscedasticity.
      \item AR($p$) is OLS with lagged $y$. Use \textbf{BIC} to select $p$.
            Multi-step forecasts use recursive substitution.
      \item Three pitfalls: \textbf{spurious regression} (I(1) levels),
            \textbf{overfitting} (too many predictors), and the
            \textbf{regressor availability problem} (data leakage).
    \end{enumerate}
  \end{keybox}
  \vspace{0.1cm}
  \muted{\footnotesize\itshape Next: Exponential Smoothing (Lecture 3) ---
  adaptive weighting without requiring explicit predictor specification.}
\end{frame}

% --- References --------------------------------------------------------------
\begin{frame}[allowframebreaks]{References}
  \printbibliography[heading=none]
\end{frame}

\end{document}
